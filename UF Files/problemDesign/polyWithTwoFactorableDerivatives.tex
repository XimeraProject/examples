\documentclass{ximera}
\title{Cubic with Two Factorable Derivatives}
%% You can put user macros here
%% However, you cannot make new environments

\graphicspath{
{./}
{graphicsAndVideos/}
}

\usepackage{tikz}
\usepackage{tkz-euclide}

\newcommand{\licenseAcknowledgement}{Licensed under Creative Commons 4.0}


\newcommand{\dfn}{\textbf}

\pgfplotsset{compat=1.7} % prevents compile error.

\tikzstyle geometryDiagrams=[ultra thick,color=blue!50!black]

\usepackage{lipsum}

\newcommand{\ximera}{XiMeRa}
%\def\d{\,d}
\renewcommand{\d}{\,d}


\newenvironment{MM}{\begin{remark}}{\end{remark}}


%\newtheorem*{MM}{Metacognitive Moment}
%\ximerizedEnvironment{MM}


%% FOR XIMERA CLS %%%%%%%%%%%%%%%%%%%%%%%%%%%%%%%

\renewcommand{\link}[2][]{\href{#2}{#1}% the optional argument,
                                       % a name for the link as a clickable link
\ifthenelse{\equal{#1}{}}% if no optional argument, then 
{\url{#2}}% just print the name
{\footnote{See #1 at \url{#2}}}}% include the link address as a footnote

%%%%%%%%%%%%%%%%%%%%%%%%%%%%%%%%%%%%%%%%%%%%%%%%%%




%% \makeatletter%% note could be made fancy with if then tech above.
%% %\setcounter{secnumdepth}{2}
%% \def\@othersection\section
%% \def\@othersubsection\subsection
%% \newcommand{\sectionstyle}{\def\section{\@othersection}\def\maketitle{\newpage \noindent {\bf A pretitle:}\\[-3em]\section\@title}\def\section{\subsection}}
%% \newcommand{\chapterstyle}{\def\maketitle{\newpage \noindent {\bf A pretitle:}\\[-3em]\chapter\@title}}
%% \makeatother


\begin{document}


\begin{abstract}
How to generate ``nice'' polynomials that factor well, along with its first and second derivative.
\end{abstract}
\maketitle



\begin{sagesilent}
def RandInt(a,b):
    """ Returns a random integer in [`a`,`b`]. Note that `a` and `b` should be integers themselves to avoid unexpected behavior.
    """
    return QQ(randint(int(a),int(b)))
    # return choice(range(a,b+1))

def NonZeroInt(b,c, avoid = [0]):
    """ Returns a random integer in [`b`,`c`] which is not in `av`. 
        If `av` is not specified, defaults to a non-zero integer.
    """
    while True:
        a = RandInt(b,c)
        if a not in avoid:
            return a



## Start with a while loop to make sure the result has reasonable coefficients, at least in general size.
p1f2 = 9999*x^3 + 9999*x^2 + 9999*x + 9999

while ((abs(p1f2.coefficient(x^3))>100) or (abs(p1f2.coefficient(x^2))>100) or (abs(p1f2.coefficient(x))>100) or (abs(p1f2(x=0))>100)):
    ### We start by taking a product of factors to get a factorable first derivative.
     
    # Make sure the leading coefficient is divisible by 3, which will help ensure the numbers stay nice(ish) after integrating.
    p1c1 = 1
    p1c3 = 1
    while mod(p1c1*p1c3,3)>0:
        p1c1 = NonZeroInt(-5,5)
        p1c2 = NonZeroInt(-5,5)
        p1c3 = RandInt(1,6)
        p1c4 = -sign(p1c1)*sign(p1c2)*RandInt(1,5)# Rigged sign to make sure we get a difference of squares in the cube, not sum.
    
     
    p1fact1 = p1c1*x-p1c2
    p1fact2 = p1c3*x-p1c4
     
    p1f1 = expand(p1fact1*p1fact2)
    
    
    
    ### Now we make the original function by integrating, then finding an appropriate ‘‘C’’ to add to make it factor by grouping in some nice way.
     
    p1f2temp = integral(p1f1,x)
     
    # Now, we assume we are going to factor by grouping, to be kind, so we extract the necessary constant we will need:
    p1c5 = p1f2temp.coefficient(x^2)*p1f2temp.coefficient(x)/p1f2temp.coefficient(x^3)
     
    p1f2 = p1f2temp+p1c5
     
    if p1f2.coefficient(x^3)*p1f2.coefficient(x)<0:
        p1sqrtval = p1f2.coefficient(x)/p1f2.coefficient(x^3)
        p1f2fact1a = x - sqrt(abs(p1sqrtval))
        p1f2fact1b = x + sqrt(abs(p1sqrtval))
        p1f2fact1 = -sign(p1sqrtval)*(p1f2fact1a)*(p1f2fact1b)#(-sqrt(-p1f2.coefficient(x^3))*x + sqrt(p1f2.coefficient(x)))*(sqrt(-p1f2.coefficient(x^3))*x + sqrt(p1f2.coefficient(x)) )
        p1f2fact2 = p1f2.coefficient(x^3)*x + p1f2.coefficient(x^2)
        p1f1zero1 = -p1f2fact1a(x=0)/p1f2fact1a.coefficient(x)
        p1f1zero2 = -p1f2fact1b(x=0)/p1f2fact1b.coefficient(x)
        p1f1zero3 = -p1f2fact2(x=0)/p1f2fact2.coefficient(x)
    else:
        p1f2fact1 = p1f2.coefficient(x^3)*x^2 + p1f2.coefficient(x)
        p1f2fact2 = x + p1f2.coefficient(x^2)/p1f2.coefficient(x^3)
        p1f1zero1 = 0
        p1f1zero2 = 0
        p1f1zero3 = -p1f2fact2(x=0)/p1f2fact2.coefficient(x)
    
    p1f2check = expand(p1f2fact1*p1f2fact2)
    
    
     
     
    ### To get the second derivative function we can take the derivative of the original - which must be linear and so we know it is easy to solve for students.
     
    p1f3 = derivative(p1f1,x)



\end{sagesilent}

\section{The Problem}

Often we want to have a polynomial whose first and second derivative are suitably ``nice'' for students in order to allow them to factor the polynomial itself, as well as its first and second derivatives, in order to find zeros for things like graphing. Unfortunately, it turns out, that creating a polynomial that is factorable, and whose first and second derivative are also factorable, is a \textit{highly} nontrivial problem. Here we provide code that works (and generates \textit{relatively} nice polynomials as well) to generate such a polynomial that is a cubic function (this also helps avoid requiring something like the rational root theorem for factoring the original polynomial - in this case the cubic is always factorable by grouping as well).



\section{The Result}

The original functions is $f(x) = \sage{p1f2} $
    which should factor into: 
    \[\sage{p1f2fact1}\left(\sage{p1f2fact2}\right)\] 

The first derivative is $f'(x) = \sage{p1f1}$ 
    which should factor into 
    \[\left(\sage{p1fact1}\right)\cdot\left(\sage{p1fact2}\right)\]

The second derivative is $f''(x) = \sage{p1f3}$ 
    which is linear so it factors trivially (Yes, we kind of cheat here).
    


%    As a check, expanding out the factored form gets me: $\sage{p1f2check}$
    
    


\section{The Magic}    
    
Here is the sagecode that generates the above problem:

\begin{verbatim}
\begin{sagesilent}
def RandInt(a,b):
    """ Returns a random integer in [`a`,`b`]. Note that `a` and `b` should be integers themselves to avoid unexpected behavior.
    """
    return QQ(randint(int(a),int(b)))
    # return choice(range(a,b+1))

def NonZeroInt(b,c, avoid = [0]):
    """ Returns a random integer in [`b`,`c`] which is not in `av`. 
        If `av` is not specified, defaults to a non-zero integer.
    """
    while True:
        a = RandInt(b,c)
        if a not in avoid:
            return a



## Start with a while loop to make sure the result has reasonable coefficients, at least in general size.
p1f2 = 9999*x^3 + 9999*x^2 + 9999*x + 9999

while ((abs(p1f2.coefficient(x^3))>100) or (abs(p1f2.coefficient(x^2))>100) or (abs(p1f2.coefficient(x))>100) or (abs(p1f2(x=0))>100)):
    ### We start by taking a product of factors to get a factorable first derivative.
     
    # Make sure the leading coefficient is divisible by 3, which will help ensure the numbers stay nice(ish) after integrating.
    p1c1 = 1
    p1c3 = 1
    while mod(p1c1*p1c3,3)>0:
        p1c1 = NonZeroInt(-5,5)
        p1c2 = NonZeroInt(-5,5)
        p1c3 = RandInt(1,6)
        p1c4 = -sign(p1c1)*sign(p1c2)*RandInt(1,5)# Rigged sign to make sure we get a difference of squares in the cube, not sum.
    
     
    p1fact1 = p1c1*x-p1c2
    p1fact2 = p1c3*x-p1c4
     
    p1f1 = expand(p1fact1*p1fact2)
    
    
    
    ### Now we make the original function by integrating, then finding an appropriate ‘‘C’’ to add to make it factor by grouping in some nice way.
     
    p1f2temp = integral(p1f1,x)
     
    # Now, we assume we are going to factor by grouping, to be kind, so we extract the necessary constant we will need:
    p1c5 = p1f2temp.coefficient(x^2)*p1f2temp.coefficient(x)/p1f2temp.coefficient(x^3)
     
    p1f2 = p1f2temp+p1c5
     
    if p1f2.coefficient(x^3)*p1f2.coefficient(x)<0:
        p1sqrtval = p1f2.coefficient(x)/p1f2.coefficient(x^3)
        p1f2fact1a = x - sqrt(abs(p1sqrtval))
        p1f2fact1b = x + sqrt(abs(p1sqrtval))
        p1f2fact1 = -sign(p1sqrtval)*(p1f2fact1a)*(p1f2fact1b)#(-sqrt(-p1f2.coefficient(x^3))*x + sqrt(p1f2.coefficient(x)))*(sqrt(-p1f2.coefficient(x^3))*x + sqrt(p1f2.coefficient(x)) )
        p1f2fact2 = p1f2.coefficient(x^3)*x + p1f2.coefficient(x^2)
        p1f1zero1 = -p1f2fact1a(x=0)/p1f2fact1a.coefficient(x)
        p1f1zero2 = -p1f2fact1b(x=0)/p1f2fact1b.coefficient(x)
        p1f1zero3 = -p1f2fact2(x=0)/p1f2fact2.coefficient(x)
    else:
        p1f2fact1 = p1f2.coefficient(x^3)*x^2 + p1f2.coefficient(x)
        p1f2fact2 = x + p1f2.coefficient(x^2)/p1f2.coefficient(x^3)
        p1f1zero1 = 0
        p1f1zero2 = 0
        p1f1zero3 = -p1f2fact2(x=0)/p1f2fact2.coefficient(x)
    
    p1f2check = expand(p1f2fact1*p1f2fact2)
    
    
     
     
    ### To get the second derivative function we can take the derivative of the original - which must be linear and so we know it is easy to solve for students.
     
    p1f3 = derivative(p1f1,x)

\end{sagesilent}

\end{verbatim}





\end{document}