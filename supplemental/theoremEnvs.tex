\documentclass{ximera}
\outcome{Theorem environments.}
%% You can put user macros here
%% However, you cannot make new environments

\graphicspath{
{./}
{graphicsAndVideos/}
}

\usepackage{tikz}
\usepackage{tkz-euclide}

\newcommand{\licenseAcknowledgement}{Licensed under Creative Commons 4.0}


\newcommand{\dfn}{\textbf}

\pgfplotsset{compat=1.7} % prevents compile error.

\tikzstyle geometryDiagrams=[ultra thick,color=blue!50!black]

\usepackage{lipsum}

\newcommand{\ximera}{XiMeRa}
%\def\d{\,d}
\renewcommand{\d}{\,d}


\newenvironment{MM}{\begin{remark}}{\end{remark}}


%\newtheorem*{MM}{Metacognitive Moment}
%\ximerizedEnvironment{MM}


%% FOR XIMERA CLS %%%%%%%%%%%%%%%%%%%%%%%%%%%%%%%

\renewcommand{\link}[2][]{\href{#2}{#1}% the optional argument,
                                       % a name for the link as a clickable link
\ifthenelse{\equal{#1}{}}% if no optional argument, then 
{\url{#2}}% just print the name
{\footnote{See #1 at \url{#2}}}}% include the link address as a footnote

%%%%%%%%%%%%%%%%%%%%%%%%%%%%%%%%%%%%%%%%%%%%%%%%%%




%% \makeatletter%% note could be made fancy with if then tech above.
%% %\setcounter{secnumdepth}{2}
%% \def\@othersection\section
%% \def\@othersubsection\subsection
%% \newcommand{\sectionstyle}{\def\section{\@othersection}\def\maketitle{\newpage \noindent {\bf A pretitle:}\\[-3em]\section\@title}\def\section{\subsection}}
%% \newcommand{\chapterstyle}{\def\maketitle{\newpage \noindent {\bf A pretitle:}\\[-3em]\chapter\@title}}
%% \makeatother

\author{Bart Snapp \and Rodney Austin}

\title{Theorem-like environments}


\begin{document}
\begin{abstract}
  Examples of the theorem environments.
\end{abstract}
\maketitle

%\section*{Theorem-Like Environments}
    \subsection*{How theorem-like environments work}
        Ximera provides a number of theorem-like environments by default. These are all functionally identical, the only difference is the name associated with them. It is important to note that each of these theorem-like environments use their own independent counter/numbering. For example, observe the numberings of the following theorems and corollaries:
        
        \begin{theorem}
            Here's the first theorem!
        \end{theorem}
        
        \begin{corollary}
            Here's the first corollary!
        \end{corollary}
    
        \begin{corollary}
            Here's the second corollary!
        \end{corollary}
    
        \begin{theorem}
            Here's the second theorem!
        \end{theorem}
        
        So the different environments will number themselves in order, but each tracks its own numbering independently.

    \subsection*{Optional Arguments}
        
        There are (currently) no Ximera-specific optional arguments - but Ximera does support the default optional arguments. For example, you can put an optional argument to provide a ``nickname'' for the theorem. For example:
        
        \begin{axiom}[Axiom of Choice]
            I'm not going to actually write the Axiom of Choice here - but only crazy people don't assume it. Or maybe non-analysts.
        \end{axiom}
        
        The above is generated with:
        \begin{verbatim}
        \begin{axiom}[Axiom of Choice]
            I'm not going to actually write the Axiom of Choice here - 
            but only crazy people don't assume it. Or maybe non-analysts.
        \end{axiom}        
        \end{verbatim}
        
        
    \subsection*{Examples}
        The easiest way to list the default theorem-like environments, is just to demonstrate them all - prepare for a massive example list!
        
        \begin{theorem}
            \lipsum[1][1-3]
        \end{theorem}
        
        
        \begin{theorem}[My theorem]
            \lipsum[1][1-3]
        \end{theorem}
        
        \begin{algorithm}
            \lipsum[1][1-3]
        \end{algorithm}
        
        \begin{axiom}
            \lipsum[1][1-3]
        \end{axiom}
        
        \begin{claim}
            \lipsum[1][1-3]
        \end{claim}
        
        \begin{conclusion}
            \lipsum[1][1-3]
        \end{conclusion}
        
        \begin{condition}
            \lipsum[1][1-3]
        \end{condition}
        
        \begin{conjecture}
            \lipsum[1][1-3]
        \end{conjecture}
        
        \begin{corollary}
            \lipsum[1][1-3]
        \end{corollary}
        
        \begin{criterion}
            \lipsum[1][1-3]
        \end{criterion}
        
        \begin{definition}
            \lipsum[1][1-3]
        \end{definition}
        
        \begin{example}
            \lipsum[1][1-3]
        \end{example}
        
        \begin{explanation}
            \lipsum[1][1-3]
        \end{explanation}
        
        \begin{fact}
            \lipsum[1][1-3]
        \end{fact}
        
        \begin{formula}
            \lipsum[1][1-3]
        \end{formula}
        
        \begin{idea}
            \lipsum[1][1-3]
        \end{idea}
        
        \begin{lemma}
            \lipsum[1][1-3]
        \end{lemma}
        
        \begin{model}
            \lipsum[1][1-3]
        \end{model}
        
        \begin{notation}
            \lipsum[1][1-3]
        \end{notation}
        
        \begin{observation}
            \lipsum[1][1-3]
        \end{observation}
        
        \begin{paradox}
            \lipsum[1][1-3]
        \end{paradox}
        
        \begin{procedure}
            \lipsum[1][1-3]
        \end{procedure}
        
        \begin{proposition}
            \lipsum[1][1-3]
        \end{proposition}
        
        \begin{remark}
            \lipsum[1][1-3]
        \end{remark}
        
        \begin{summary}
            \lipsum[1][1-3]
        \end{summary}
        
        \begin{template}
            \lipsum[1][1-3]
        \end{template}
        
        \begin{warning}
            \lipsum[1][1-3]
        \end{warning}

    \subsection*{Best Practices}
        
        
        
    \subsection*{Potential Pitfalls and Problems}
        \subsubsection*{Page Credit}
            Remember that environments are what generate page credit in Ximera. So if you include one of these environments, and don't have anything which requires student input inside that environment (such as a multiple choice environment or an answer command), it will still count as credit for the page, and immediately grant student credit for that environment as it will immediately be marked as ``completed'' (since there is nothing for the student to do). This can be a good thing - to use as partial credit for at least opening the assignment for instance. See the segment on Credit Allocation for more on this issue in general.
        \subsubsection*{Cross-Referencing}
            One of LaTeX's strong points is automatically numbering and cross-referencing content via a label and ref command. This still works in the generated pdf (at time of writing), but it does \textbf{not} work in any meaningful way online. In general, one is best served by using a hard-coded reference and possibly a link command to reference another part of the document, to make sure that it works online as well as online. See the advanced section on linking content for more on this. 
        
        
\end{document}
