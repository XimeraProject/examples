\documentclass{ximera}
\title{Footnotes}


\begin{document}
\begin{abstract}
    Show that footnotes work without anything special.
\end{abstract}
\maketitle

%\section*{Footnotes}
    \subsection*{How Footnotes Work Online}
        Footnotes are used frequently in math text, but it isn't clear how they should be dealt with in an online format. Eventually I aim to build out footnotes to work more like \href{https://what-if.xkcd.com/}{the xkcd what-if footnotes}. For now though, footnotes \textit{do} work, and they work by expanding in-line in a different color font.
        
        So, you can use something like:
        
        \begin{verbatim}
            We love to have footnotes in math texts%
            \footnote{Or, at least, they seem to keep getting used...}
            even though they make you go look elsewhere to see what is happening.
        \end{verbatim}
        
        Becomes:
        
        We love to have footnotes in math texts%
        \footnote{Or, at least, they seem to keep getting used...}
        even though they make you go look elsewhere to see what is happening.
        
        Thus there is nothing special about the syntax or usage to let you use footnotes. It is worth a note here though that you may need to update your Ximera.cls, Ximera.4ht, and possibly a few other files to make this work - but if you are interested in doing so it's a simple matter of copy/pasting the files, just contact the Ximera dev and they will send you updated files.


        
    \subsection*{Optional Arguments}
        There are currently no optional arguments for these environments.

    \subsection*{(Another) Examples}
    
        Do we need another example?\footnote{Apparently We Do.}
        
    \subsection*{Best Practices}
    
        \subsubsection*{Consistency is important}
            Much like the foldable/expandable content, footnotes should try to capture a specific type of content, and be relatively consistent throughout its usage. You may use footnotes to include parenthetical information, or references to further/deeper information, or funny joke asides. What you don't want is to have students get use to expecting that footnotes are amusing little jokes, and then have one that contains absolutely vital information halfway through the semester, where some students may have started skipping the footnotes\footnote{Maybe they don't like your sense of humor!}, as they may then feel like they were tricked.
        
    
    \subsection*{Potential Pitfalls and Problems}
        \subsubsection*{Footnote expansion}
            Footnote expansion online is done in-line, which can make formatting a little weird if you include a long enough footnote - or LaTeX commands - that generate paragraph breaks. The footntoe should expand naturally across multiple lines, but actual paragraph breaks can make sentence structuring weird - especially if the footntoe is in the middle of a sentence, rather than at the end. This doesn't generate any technical issues - everything will still validate and render correctly - but it can be visually jarring to a student.

        \subsubsection*{Content is only hidden, but it exists}
            Technically the content within a footnote command exists, but is just hidden until triggered. This means, if you want to do something like put a problem within the footnote command, and a student completes the page without triggering the footnote, then the problem they don't see still counts for ``page credit'' meaning that the student won't have full page credit, but won't see \textit{why} they don't have full page credit.
            
            
            
\end{document}