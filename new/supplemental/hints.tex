\documentclass{ximera}
\title{Hints}


\begin{document}
\begin{abstract}
    Demonstration of the 'hints' feature.
\end{abstract}
\maketitle

\section*{Hints}
    \subsection*{How to generate Hints}

        Hints can be added via the hint environment. This provides an unfoldable hint that students can click to get a hint for their problem. Consider the following:
        
        \begin{problem}
            \begin{hint}
                The answer is 42
            \end{hint}
            What number am I thinking of? $\answer{42}$
        \end{problem}
        
        Notice that the hint requires a certain amount of time (30 seconds) on the page before a student can open it - to stop students from just spamming the ``hint'' button to get enough hints to make the problem trivial.
        
        You can also add multiple hints to a problem, consider:
        
        \begin{problem}
            \begin{hint}
                The answer is prime...
            \end{hint}
            \begin{hint}
                The answer is not even.
            \end{hint}
            \begin{hint}
                The answer is less than 5.
            \end{hint}
            What number am I thinking of? $\answer{3}$
        \end{problem}
        
        The above is generated by:
        
        \begin{verbatim}
            \begin{problem}
                \begin{hint}
                    The answer is prime...
                \end{hint}
                \begin{hint}
                    The answer is not even.
                \end{hint}
                \begin{hint}
                    The answer is less than 5.
                \end{hint}
                What number am I thinking of? $\answer{3}$
            \end{problem}
        \end{verbatim}
        
        Notice that, if you want to have multiple hints, it's best to not make them nested, that way the hint counter will correctly know how many hints there are when displaying the button to click.
        
        
    \subsection*{Optional Arguments}
        
        There are currently no optional arguments for the hint environment.

    \subsection*{Examples}
    
        TBD
        
    \subsection*{Best Practices}
    
        TBD
        
    \subsection*{Potential Pitfalls and Problems}
        
        None known, but this hasn't been used much... or at all really.

\end{document}