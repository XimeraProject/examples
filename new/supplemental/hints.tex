\documentclass{ximera}
\title{Hints}


\begin{document}
\begin{abstract}
    Demonstration of the 'hints' feature.
\end{abstract}
\maketitle

\section*{Hints}
    \subsection*{How to generate Hints}

        Hints can be added via the hint environment. This provides an unfoldable hint that students can click to get a hint for their problem. Consider the following:
        
        \begin{problem}
            \begin{hint}
                The answer is 42
            \end{hint}
            What number am I thinking of? $\answer{42}$
        \end{problem}
    
    \subsection*{Access Delay}
        Notice that the hint requires a certain amount of time (30 seconds) on the page before a student can open it - to stop students from just spamming the ``hint'' button to get enough hints to make the problem trivial.
        
        You can also add multiple hints to a problem, consider:
        
        \begin{problem}
            \begin{hint}
                The answer is prime...
            \end{hint}
            \begin{hint}
                The answer is not even.
            \end{hint}
            \begin{hint}
                The answer is less than 5.
            \end{hint}
            What number am I thinking of? $\answer{3}$
        \end{problem}
        
        The above is generated by:
        
        \begin{verbatim}
            \begin{problem}
                \begin{hint}
                    The answer is prime...
                \end{hint}
                \begin{hint}
                    The answer is not even.
                \end{hint}
                \begin{hint}
                    The answer is less than 5.
                \end{hint}
                What number am I thinking of? $\answer{3}$
            \end{problem}
        \end{verbatim}
        
        Notice that, if you want to have multiple hints, it's best to not make them nested, that way the hint counter will correctly know how many hints there are when displaying the button to click.
        
        
    \subsection*{Optional Arguments}
        
        There are currently no optional arguments for the hint environment.

    \subsection*{(Additional) Examples}
    
        See the above example. Not sure if we need another example here?
        
    \subsection*{Best Practices}
    
        \subsubsection*{Feedback versus Hints}
            There are (currently) two canonical ways to provide supporting information to the student while they work on a problem, either feedback or hints. As a purely anecdotal - but shockingly consistent and ubiquitous - observation, students view feedback as something that (at best) is a followup once the problem is correct, whereas ``hints'' are used to help get to understand how to complete a problem. 
            
            For a precalculus course, I included feedback to help students understand how to complete problems, on every single problem. At the end of the first semester after doing this, in a course of 500+ students, over half of them (just over 300) requested more support for the online problems/homework, despite every problem having feedback to help them figure out how to complete the problem. 
            
            For the next semester, I renamed the feedback environments into hint environments and moved them to show up above the problem statement instead of below. I didn't change the content of the environment, nor did I add any content to any of the environments. At the end of that semester, out of around 450 students, I had less than 10 percent (around 30) students request additional support. Just changing from feedback to hint had a \textit{massive} impact on the student feedback, even though the content was unchanged. 
            
            This could be a population quirk due to something about students attending my university (University of Florida) but I suspect it has more to do with how online educational resources have conformed and built expectations among students as my students are almost entirely first year university students.
        
        \subsubsection*{Elevating hints}
            You can provide multiple hints to allow students to get subsequent hints in case the first hint isn't sufficient. Most students seem to understand that they should try to complete the problem with the minimal amount of hints used, but many will spam-open the hints as fast as possible. For this reason, it can be useful to provide detailed hints to one problem, then fewer, or even no hints to followup similar problems. 
        
        \subsubsection*{Hints can reflect the problem itself}
            It is often helpful to have the hints contain the same values as the problem they are hints for. This can even be done when using generated/randomized problems via something like sage, which allows the hints to be dynamic along with the problems. This often invalidates the difficulty of the specific problem the hints are for, as the hints tend to then reveal too much of the problem structure and/or calculations. Nonetheless this can be viewed as a sort of ``worked out solution'' to a specific problem that the student can refer to before doing other similar problems without these kinds of hints. This has pedagogical pros and cons, and whether it is indeed ``best practice'' or not is more dependent on the student's mathematical maturity and level, and the context of the course and the problem, than it is about the generic notion of hints - so it is included here more as an idea to consider by the content author, to see if it is an appropriate approach for their setting and students.
        
    \subsection*{Potential Pitfalls and Problems}
        
        \subsubsection*{Setting Expectations}
            More of a course issue than an issue with the hint environment - but remember to keep in mind what expectations are being set by your hints. Too many hints, especially early on, can generate student complaints if later in the semester, the problems have fewer hints. Moreover, excessive detail or quantity of hints can generate false confidence for a student, leading to poor performance on exams or quizzes, and the feeling like there was a considerable ``jump in difficulty'' between the Ximera work and the exams/quizzes/etc.
            
            
        \subsubsection*{Content is only hidden, but it exists}
            Technically the content within a hint environment exists, but is just hidden until triggered. This means, if you want to do something like put another problem within the hint environment, and a student completes the page without triggering the hint, then the problem they don't see still counts for ``page credit'' meaning that the student won't have full page credit, but won't see \textit{why} they don't have full page credit.
            

\end{document}