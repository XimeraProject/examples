\documentclass{ximera}
\title{YouTube Embeddings}


\begin{document}
\begin{abstract}
    How to embed YouTube videos, and how they are validated.
\end{abstract}
\maketitle

%\section*{Embedding YouTube}
    \subsection*{How Embedding YouTube Works}
        YouTube content can be embedded (when embedding that video is allowed by the video's owner) easily in Ximera by using the \verb|\youtube| command. The command has most of the embedding and url information already built in, so the only thing you need to include is the part of the url that comes after the ``v='' in the url.
        
        For example, in order to embed the video found at the url: \url{https://www.youtube.com/watch?v=FvgF95i0_lw} you would use the command \verb|\youtube{FvgF95i0_lw}| which would embed the video into the page, like this:
        
        \youtube{FvgF95i0_lw}
    
    
    \subsection*{How is it validated?}
        YouTube videos count toward the progress of a tile, but it only checks to see what the saved progress of watching the video is. So if a student watches the video all the way to the end, then they would get full credit for that segment of their grade (see the previous tile on \href{https://xronos.clas.ufl.edu/examples/exampleCore/assignments/creditAllocation}{how credit broken up and assigned/earned}). 
        
        
    \subsection*{Optional arguments}
        There are no optional arguments for the youtube command at this point. 
    

    \subsection*{(Additional) Examples}
    
        Do we need more examples?
        
    \subsection*{Best Practices}
    
        \subsubsection*{Students Expect Text to be Redundant}
            In settings where you have, for example, a lecture video as well as a textual explanation, (e.g. a textbook chapter with a lecture video embedded), students tend to expect that these two things are saying \textit{exactly} the same kind of thing. I have gotten significant pushback from students for having different examples worked out in text than in lecture videos on the same page. For this reason, it is ideal to make it clear to students explicitly - and early on - what the role of your embedded videos are fulfilling. In particular, whether it is sufficient to only watch video, or if they must also read all the text, or if they can skip the video and only read text, etc. Otherwise the default expectation will likely catch you by surprise and generate student malcontent.
            
            % Author comment from Jason N: I've actually had students complain to the dept chair because I ``had the audacity in this day and age, to ask about content contained in the text, thus forcing [the student] to actually READ the text, instead of only watching the video.''
            
        \subsubsection*{There are a lot of YouTube Videos out there...}
            In most cases, you are probably going to use video content you create yourself, but sometimes there are just really great explanations or relevant videos by other content producers on YouTube already. Keep in mind that those videos are still technically copyrighted. There are a number of issues around IP laws and educational fair use, but the reality is that most educational YouTube content authors are happy to let you embed their content, as long as you do it correctly (which this command does), so it makes life much simpler to just contact them for permission.
        
    \subsection*{Potential Pitfalls and Problems}
        \subsubsection*{Progress is tied to time stamp of last watched}
            It is noteworthy that the credit gained for a video is tied to the percentage last completed of the video. This has two main points to consider.
            \begin{enumerate}
                \item If the student skips to the end of the video, they can immediately get credit for watching the video, even though they didn't watch it. This information is captured - so we plan on allowing or disallowing this ability in the future, and possibly having a notification for when students skip ahead for credit instead of watching video.
                \item Sometimes YouTube will stop the video a second or two before the actual end of the video. We implemented a ``good enough'' feature to video watching to account for this - basically if they watch more than 95\% or so of the video it counts as full credit, but depending on the length of the video, this can sometimes still have a video registered as not completed by enough to mark the tile as not fully completed. This, in turn, can cause students to lose their minds as they try to assure their instructor that they watched everything fully and did everyone on the page, but can't get full credit.
            
            \end{enumerate}

\end{document}