\documentclass{ximera}
%% You can put user macros here
%% However, you cannot make new environments

\graphicspath{
{./}
{graphicsAndVideos/}
}

\usepackage{tikz}
\usepackage{tkz-euclide}

\newcommand{\licenseAcknowledgement}{Licensed under Creative Commons 4.0}


\newcommand{\dfn}{\textbf}

\pgfplotsset{compat=1.7} % prevents compile error.

\tikzstyle geometryDiagrams=[ultra thick,color=blue!50!black]

\usepackage{lipsum}

\newcommand{\ximera}{XiMeRa}
%\def\d{\,d}
\renewcommand{\d}{\,d}


\newenvironment{MM}{\begin{remark}}{\end{remark}}


%\newtheorem*{MM}{Metacognitive Moment}
%\ximerizedEnvironment{MM}


%% FOR XIMERA CLS %%%%%%%%%%%%%%%%%%%%%%%%%%%%%%%

\renewcommand{\link}[2][]{\href{#2}{#1}% the optional argument,
                                       % a name for the link as a clickable link
\ifthenelse{\equal{#1}{}}% if no optional argument, then 
{\url{#2}}% just print the name
{\footnote{See #1 at \url{#2}}}}% include the link address as a footnote

%%%%%%%%%%%%%%%%%%%%%%%%%%%%%%%%%%%%%%%%%%%%%%%%%%




%% \makeatletter%% note could be made fancy with if then tech above.
%% %\setcounter{secnumdepth}{2}
%% \def\@othersection\section
%% \def\@othersubsection\subsection
%% \newcommand{\sectionstyle}{\def\section{\@othersection}\def\maketitle{\newpage \noindent {\bf A pretitle:}\\[-3em]\section\@title}\def\section{\subsection}}
%% \newcommand{\chapterstyle}{\def\maketitle{\newpage \noindent {\bf A pretitle:}\\[-3em]\chapter\@title}}
%% \makeatother

\title{isPositive}
\begin{document}
\begin{abstract}
    This demonstrates a very basic custom validator, which returns correct if the submitted answer is positive.
\end{abstract}
\maketitle

\begin{javascript}
// NOTE: The below are intended to be used inside an \answer optional argument with the validator key, NOT in a validator environment.

var x
// Check to see if input is positive.
  function isPositive(number) {
    return number > 0;
  };

\end{javascript}

%%%%
\begin{problem}
    This shows a basic ``isPositive'' validator, that simply checks to see if the input in positive.\\
    Notice that the validator completely overrides the expected answer, as the correctness of the answer is determined by the validator returning true/false, and this validator doesn't actually use the provided answer in any way.
    
    
    Enter a number! $\answer[validator=isPositive]{22}$
    
\end{problem}


\end{document}
