\documentclass{ximera}
\title{Using Sage}


\begin{document}
\begin{abstract}
    We discuss aspects of Sage we use, and provide some core sage functions we use.
\end{abstract}
\maketitle

\section{What is Sage?}
    Sage - officially ``sagemath'' now - is an offshoot of python - it is basically python for mathematics. This means that, if you are at all familiar with python, writing sage is basically the same - with some additional commands for math specific operations and context.

\section{Using Sage Locally}
    \subsection{Where to get Sage?}
        You can read how to \href{https://www.sagemath.org/download.html}{get and install sage here}. 
    
    \subsection{How to use Sage}
        In order to use sage locally you would want to download it (see above) and then install the sagetex LaTeX package, which links the two systems together. This is relatively straight forward - the sage installation comes with the relevant LaTeX package automatically, so it's largely a process of copy/pasting the .sty file to the right location and making it available to your LaTeX installation. There are \href{https://doc.sagemath.org/html/en/reference/misc/sagetex.html}{detailed instructions here}.
    
    \subsection{How to Compile LaTeX with Sage}
        Once you have LaTeX and Sage installed locally, you'll want to setup your LaTeX editor to compile using sage as well. If your LaTeX file is named something like ``foo.tex'' then you'll want to run the following commands to compile:
        
        \begin{verbatim}
            pdflatex foo.tex
            sage foo.sagetex.sage
            pdflatex foo.tex
        \end{verbatim}
        Depending on your LaTeX editor, there is usually a way to automate this as a command, but that is specific to your editor of choice.
    
\section{Using Sage Non-Locally}
    
    If you don't want to use sage locally, you can't get sagetex working in any of the typical online editor (for example, overleaf does not support sagetex). Nonetheless you can still write the sage code and the LaTeX code and largely test them independently - since sagetex really does run sage on a generated file that is, more or less, equivalent to the sage content you included in your file.
    
    \subsection{Testing Sage Code}
        You can test sage code in any public sage compiler - usually referred to as ``sage cells''. For example, you can \href{https://sagecell.sagemath.org/}{use the official sage cell server} and copy/paste your code in with a little debug code to see if it compiles and what it yields.

\section{If you want to publish/edit locally}
    If you are publishing/editing content locally, you almost certainly want to have sage installed locally so you can properly and fully debug the sage code before trying to publish it. That being said, technically you don't need to be able to compile the sage code locally in order to publish sage code online via Ximera, since the sage is handled in another way - although it might be difficult to get the various sage LaTeX commands to compile during the ``bake'' process.

\end{document}