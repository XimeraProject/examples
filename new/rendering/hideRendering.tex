\documentclass{ximera}
\title{Disable Rendering}


\begin{document}
\begin{abstract}
    A description of how to disable or hide rendered material - while still executing that code.
\end{abstract}
\maketitle
   
\section*{How to Suppress Content from Rendering.}
    \subsection*{Mechanism of Rendering}
        
        There are a number of ways to do this. Probably the best option is to use the commands pdfOnly, or the environments prompt or onlineOnly. The prompt environment and pdfonly command functionally do the same thing - they display their contents in the pdf that is compiled, but they do not display their content in the online webpage. For example, the following should appear blank online:
        
        \begin{prompt}
            This is the contents of a prompt environment.
        \end{prompt}
        
        However, above - in the pdf - will be the ``This is the content of a prompt environment'' because we used the following code:
        
        \begin{verbatim}
            \begin{prompt}
                This is the contents of a prompt environment.
            \end{prompt}
        \end{verbatim}

        Similarly the following should be blank online: \pdfOnly{some random content - but this command is usually used to call commands or set configurations that apply only to the pdf output}. But, in the pdf there will be a line of text, since we used the command:\\
        
        \begin{verbatim}
        \pdfOnly{some random content - but this command is usually used to 
        call commands or set configurations that apply only to the pdf output}
        \end{verbatim}
        
        Finally, you can get stuff to compile to the online version, but omit it from the pdf, by using the onlineOnly environment. For example:
        
        \begin{onlineOnly}
            This should only be visible online. At least in theory.
        \end{onlineOnly}
        
        This was generated using the code:
        
        \begin{verbatim}
            \begin{onlineOnly}
                This should only be visible online. At least in theory.
            \end{onlineOnly}
        \end{verbatim}
    
    \subsection*{Any Necessary Content}
    
    
    
    \subsection*{Quirks of Rendering}
    
    
    
    \subsection*{Any Ximera-Specific Optional Arguments}
    
    
    
    \subsection*{Accessibility}
    
    
    
    \subsection*{Potential Problems and Pitfalls}
    
    
    
    \subsection*{Any Best Practices or Advice}
    
    
    
\end{document}