\documentclass{ximera}
\title{Answer and PDFs}

\begin{document}
\begin{abstract}
    A description of how the answer box behaves in the compiled pdf
\end{abstract}
\maketitle

%\section*{The answer box}

    \subsection*{Answer behavior in the PDF}

        In the PDF, the default behavior is showing all answers and solutions. Then \verb|$\answer{x^2 + 3x + 1}$| will display the answer as \fbox{$x^2 + 3x + 1$} in the PDF.
                
        Using \verb|\documentclass[handout]{ximera}| suppresses the answers. 
        
        In PDF \verb|$\answer{x^2 + 3x + 1}$| is by default rendered as {\handoutfalse $\answer{x^2 + 3x + 1}$}, and in handout mode normally as {\handouttrue $\answer{x^2 + 3x + 1}$},
        but with \verb|$\answer[given]{x^2 + 3x + 1}$| as {\handouttrue $\answer[given]{x^2 + 3x + 1}$}. 
        \begin{onlineOnly}
            Note: in order to make sense of this, you'll have to consult the PDF version of this manual!
        \end{onlineOnly}

        This behavior can be redefining by setting the \verb|\defaultAnswerFormat|, \verb|\handoutAnswerFormat| and \verb|\givenAnswerFormat| commands. 

        \begin{exercise} Only in the PDF interesting things can be seen. Online these settings should not have any effect.
            \newcommand{\mytmptable}{
                \begin{tabular}{ll}
                long: & $\answer{1234567890}$,  and with [given]: $\answer[given]{1234567890}$. 
                \\
                short:& $\answer{15}$, and with [given]: $\answer[given]{12}$.   
                \end{tabular}
              }
            
              The format of \verb|\answer| can be adapted. Following code
              %
              \begin{verbatim}
                \begin{tabular}{ll}
                  long: & $\answer{1234567890}$, and with [given]: $\answer[given]{1234567890}$. 
                  \\
                  short:& $\answer{15}$,         and with [given]: $\answer[given]{12}$.   
                \end{tabular}
              \end{verbatim}
              gives by default :
              
              \handouttrue Handoutmode: \mytmptable
            
              \handoutfalse Default mode: \mytmptable
            
              But could e.g. be changed to (handout: line, default: boxed, and given: plain)
            
              \let\handoutAnswerFormat\answerFormatFlexibleBox
              \let\defaultAnswerFormat\answerFormatBoxed
              \let\givenAnswerFormat\answerFormatPlain
            
              \handouttrue Handoutmode: \mytmptable
            
              \handoutfalse Default mode: \mytmptable
        \end{exercise}
        
    \subsection*{Examples}
        \begin{description}
            \item[(Auto)Scale Box to (Correct) Answer Size:] The answer box scales with the size of the (correct) answer. To do this, include the command 
            \begin{verbatim}
\renewcommand{\handoutAnswerFormat}[1]{
    \fbox{\scalebox{2}{\phantom{$#1$}}}}
            \end{verbatim}
            in the preamble. 
            
            Now \verb|$\answer{x^2 + 3x + 1}$| will display as \fbox{\scalebox{2}{\phantom{$x^2 + 3x + 1$}}} and \verb|$\answer{2x}$| will display as \fbox{\scalebox{2}{\phantom{$2x$}}}.
                
            \item[(Auto)Scaling Horizontal Line:] A horizontal line that scales with the size of the answer. To do this, include the command 
            \begin{verbatim}
\renewcommand{\handoutAnswerFormat}[1]{\protect
    \rule{\widthof{$#1$}*2}{0.4pt}}
            \end{verbatim}
            in the preamble. The \verb|*2| stretched the length by two to accomodate handwriting.

            Now \verb|$\answer{x^2 + 3x + 1}$| will display as
            \rule{\widthof{$x^2 + 3x + 1$}*2}{0.4pt} and \verb|$\answer{2x}$| will display as \rule{\widthof{$2x$}*2}{0.4pt}.

            \item[Adjustable Length Horizontal Line:] A horizontal line whose length can be changed for each problem. To default to line of length 2 centimeters, include 
            \begin{verbatim}
\renewcommand{\handoutAnswerFormat}[2][2cm]{\protect
    \rule{#1}{0.4pt}}
            \end{verbatim}
            in the preamble. Then \verb|$\answer{x^2 + 3x + 1}$| is   will display as \rule{2 cm}{0.4pt} while \verb|$\answer[4cm]{x^2 + 3x + 1}$| will display as \rule{4cm}{0.4pt}.
        \end{description}
    
                

    
\end{document}