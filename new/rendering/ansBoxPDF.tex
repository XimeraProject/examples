\documentclass{ximera}
\title{Answer and PDFs}

\begin{document}
\begin{abstract}
    A description of how the answer box behaves in the compiled pdf
\end{abstract}
\maketitle

%\section*{The answer box}

    \subsection*{Answer behavior in the PDF}

        In the PDF, the default behavior is showing all answers and solutions. 
        %Then \verb|$\answer{x^2 + 3x + 1}$| will display the answer as \fbox{$x^2 + 3x + 1$} in the PDF.
                
        Using \verb|\documentclass[handout]{ximera}| suppresses the answers. 
        
        In the PDF version, \verb|$\answer{x^2 + 3x + 1}$| is by default rendered as {\handoutfalse $\answer{x^2 + 3x + 1}$}, and in handout mode normally as {\handouttrue $\answer{x^2 + 3x + 1}$}.
        With \verb|$\answer[given]{x^2 + 3x + 1}$| 
        you get by default {\handoutfalse $\answer[given]{x^2 + 3x + 1}$}, 
        but in handout mode {\handouttrue $\answer[given]{x^2 + 3x + 1}$}. 

        \begin{onlineOnly}
            Note: in order to make sense of this, you'll have to consult the PDF version of this manual!
        \end{onlineOnly}

        This behavior can be changed by setting \verb|\defaultAnswerFormat|, \verb|\handoutAnswerFormat| and \verb|\givenAnswerFormat|. 

        \begin{example} Note: only in the PDF interesting things can be seen. Online these settings should not have any effect.

            \newcommand{\mytmptable}{
                \begin{tabular}{ll}
                long: & $\answer{1234567890}$ and [given]: $\answer[given]{1234567890}$. \\
                short:& $\answer{15}$  and [given]: $\answer[given]{12}$.   
                \end{tabular}
              }
            
              The format of the output of \verb|\answer| can be adapted. Following code
              %
              \begin{verbatim}
\begin{tabular}{ll}
  long: & $\answer{1234567890}$, and [given]: $\answer[given]{1234567890}$. \\
  short:& $\answer{15}$,         and [given]: $\answer[given]{12}$.   
\end{tabular}
              \end{verbatim}
              gives by default :
              
              \handouttrue Handout: \mytmptable
            
              \handoutfalse Default: \mytmptable
            
              But could e.g. be changed to (handout: flexible box, default: boxed, and given: plain)  with 

              \let\handoutAnswerFormat\answerFormatFlexibleBox
              \let\defaultAnswerFormat\answerFormatBoxed
              \let\givenAnswerFormat\answerFormatPlain

              \begin{verbatim}
\let\handoutAnswerFormat\answerFormatFlexibleBox
\let\defaultAnswerFormat\answerFormatBoxed
\let\givenAnswerFormat\answerFormatPlain
              \end{verbatim}
            
              \handouttrue Handout: \mytmptable
            
              \handoutfalse Default: \mytmptable

              or also to (handout: fixed length line, default: blue, and given: blue)  with 

              \renewcommand{\answerFormatLength}{3cm}
              \let\handoutAnswerFormat\answerFormatLine
              \let\defaultAnswerFormat\answerFormatBlue
              \let\givenAnswerFormat\answerFormatBlue

              \begin{verbatim}
\renewcommand{\answerFormatLength}{3cm}
\let\handoutAnswerFormat\answerFormatLine
\let\defaultAnswerFormat\answerFormatBlue
\let\givenAnswerFormat\answerFormatBlue
              \end{verbatim}
            
              \handouttrue Handout: \mytmptable
            
              \handoutfalse Default: \mytmptable
        \end{example}
        
    \subsection*{Provide your own formats}

        Further formats can be implemented as follows:

        \begin{description}
            \item[(Auto)Scale Box to (Correct) Answer Size:] The answer box scales with the size of the (correct) answer. To do this, include the command 
            \begin{verbatim}
\renewcommand{\handoutAnswerFormat}[1]{
    \fbox{\scalebox{2}{\phantom{$#1$}}}}
            \end{verbatim}
            in the preamble. 
            
            Now \verb|$\answer{x^2 + 3x + 1}$| will display as \fbox{\scalebox{2}{\phantom{$x^2 + 3x + 1$}}} and \verb|$\answer{2x}$| will display as \fbox{\scalebox{2}{\phantom{$2x$}}}.
                
            \item[(Auto)Scaling Horizontal Line:] A horizontal line that scales with the size of the answer. To do this, include the command 
            \begin{verbatim}
\renewcommand{\handoutAnswerFormat}[1]{\protect
    \rule{\widthof{$#1$}*2}{0.4pt}}
            \end{verbatim}
            in the preamble. The \verb|*2| stretched the length by two to accomodate handwriting.

            Now \verb|$\answer{x^2 + 3x + 1}$| will display as
            \rule{\widthof{$x^2 + 3x + 1$}*2}{0.4pt} and \verb|$\answer{2x}$| will display as \rule{\widthof{$2x$}*2}{0.4pt}.

%             \item[Adjustable Length Horizontal Line:] A horizontal line whose length can be changed for each problem. To default to line of length 2 centimeters, include 
%             \begin{verbatim}
% \renewcommand{\handoutAnswerFormat}[2][2cm]{\protect
%     \rule{#1}{0.4pt}}
%             \end{verbatim}
%             in the preamble. Then \verb|$\answer{x^2 + 3x + 1}$| is   will display as \rule{2 cm}{0.4pt} while \verb|$\answer[4cm]{x^2 + 3x + 1}$| will display as \rule{4cm}{0.4pt}.

%             % mmm, this does NOT work:
%             % ./ansBoxPDF.tex:93: Package xkeyval Error: `4cm' undefined in families `answer'.
%             % See the xkeyval package documentation for explanation.
%             \renewcommand{\handoutAnswerFormat}[2][2cm]{\protect\rule{#1}{0.4pt}}
%             \color{red}Test: \verb|$\answer[4cm]{x^2 + 3x + 1}$| gives {\handouttrue $\answer[4cm]{x^2 + 3x + 1}$}
%             %}

        \end{description}
    
                

    
\end{document}