\documentclass{ximera}
\title{multipleChoice}


\begin{document}
\begin{abstract}
    Multiple Choice Answer style syntax, usage, and known issues 
\end{abstract}
\maketitle

%\section*{Multiple Choice}
    \subsection*{How multiple choice works}
    
    
        \subsubsection*{How to generate multiple choice answers}
        
            The multiple choice answer set is generated by an environment, cleverly named 'multipleChoice' along with the choice command to enumerate choices. Indeed, the multiple choice environment is secretly a kind of retooled itemize/enumerate environment and the choice command is secretly a retooled item command. So the following:
            \begin{verbatim}
                \begin{multipleChoice}
                    \choice{First Choice is Wrong.}
                    \choice{Second Choice is Wrong.}
                    \choice{Third Choice is Wrong.}
                    \choice[correct]{Fourth Choice is Right!}
                    \choice{Fifth Choice is Wrong.}
                \end{multipleChoice}
            \end{verbatim}
            Results in:
            \begin{multipleChoice}
                \choice{First Choice is Wrong.}
                \choice{Second Choice is Wrong.}
                \choice{Third Choice is Wrong.}
                \choice[correct]{Fourth Choice is Right!}
                \choice{Fifth Choice is Wrong.}
            \end{multipleChoice}
            
            Notice that the correct answer has the optional argument [correct], more on that below.
            
            
        \subsubsection*{How multiple choice validates an answer}
        
            Simply. Multiple choice allows the student to click on (only) one answer choice, and then hit the ``check work'' button to check their answer. If that answer is marked as correct (has the ``[correct]'' optional argument), then they get credit for the problem, otherwise it is marked wrong, which greys out the answer slightly and asks the student to try again (thus the student can see which options they have already tried).
            
            Note that, this means if you have multiple choices flagged as correct, then if a student selects \textit{any} answer that is marked correct and checks it, they will get the answer marked correct - which also locks the problem, so they won't be able to check other options for correctness.
            
            
    
    \subsection*{Optional Arguments}
    
        Multiple Choice itself (currently) has no optional arguments. The choice command (currently) has only one key-value pair, which is the flag for whether the choice is correct or not. Technically the key is 'correct' and the value is 'correct' or 'incorrect', but the key defaults to 'incorrect' and simply using the key flips it to correct. In other words using \verb|\choice[correct]{Correct!}| is the same as using \verb|\choice[correct=true]{Correct!}|, and both of those will mark the given choice as the correct answer. Likewise \verb|\choice{False!}| and \verb|\choice[correct=false]{False!}| are equivalent, and both result in the answer being incorrect.
        
    \subsection*{(Additional) Examples}
    
        \subsubsection*{Example 1}
            
            \begin{problem}
                Select a prime number.
                \begin{multipleChoice}
                    \choice{1}
                    \choice[correct]{2}
                    \choice[correct]{3}
                    \choice{4}
                    \choice[correct]{5}
                \end{multipleChoice}
            \end{problem}
            
            Note that there are several potentially correct options in the above problem - but the student is only able to select one answer before submitting. As long as the student selects one of the prime numbers however, it will be marked as correct.
            
    \subsection*{Best Practices}
    
        \subsubsection*{Student Expectations}
            Despite what is demonstrated in example 1 above, it is worth a note that students have an expectation that the number of selectable answers implies the number of correct answers. In other words, since the multipleChoice environment only allows the student to select one choice, they expect then that there must only be one correct answer. For this reason, if you provide more than one correct answer, you should also include a hint, feedback, or contextual clue to students to inform them that there were multiple correct answers and you intended them to only select one of them. See the potential pitfalls/problems below for more on this.
            
        
        
    \subsection*{Potential Pitfalls and Problems}
        
        \subsubsection*{Only one answer can be selected}
            Since only one answer can be selected (selecting another one automatically deselects any previous one) then having multiple 'correct' answers in the list means a student can only select one of them. Since Ximera locks the answer once it is correct, the only way to test other answers for correctness is to ``erase work'' and start over with a different selection. Since student's won't do this in practice, and since they expect that only one answer being selectable means there is only one correct answer - this can lead to the assumption that since they got ``the'' correct answer, then all other answers \textbf{must be} incorrect, which may result in questions as to why the other (actually correct) answers aren't correct - or even worse, the blanket assumption that they must be incorrect without any questions or opportunities to correct that misunderstanding.
            
        \subsubsection*{There are a finite number of options}
            Since Ximera allows unlimited attempts at answering (this is a feature we intend to make controllable in some way in the future, but currently all answers get unlimited attempts), a clever-enough-to-be-dangerous student will realize they can just spam the answer until they find the right one and move on without even reading the options. There isn't much to do about this, but it is important to keep in mind when designing problems and deciding on answer-type variety.
            
        \subsubsection*{Randomization is 'kind of' pointless}
            Although there are ways to randomize the content of the problems and answers (see the randomization section), it is important to note that LaTeX level content \textit{cannot} be randomized currently (this is going to be changed in the future edition) which means, although the content of the choice command might be randomized, \textit{which} multiple choice option is correct \textit{cannot} be randomized - something that the students will pick up on mighty quick. For review this isn't \textit{as bad} since students are at least aware enough that they are trying to understand how the correct answer is the correct answer - so knowing ``the right answer is the third choice'' is \textit{less} important... but it's still important.
            
            It is worth noting that there \textit{is} a workaround for this that we have developed, which is \href{https://xronos.clas.ufl.edu/examples/exampleCore/problemDesign/multipleChoice}{addressed in the randomization section}, so if you want to make a randomized multiple choice problem, you should followup there.
            
        
    

\end{document}





