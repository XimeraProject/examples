\documentclass{ximera}
\title{answerBox}


\begin{document}
\begin{abstract}
    A description of how the answer box works, and it's optional arguments.
\end{abstract}
\maketitle

\section*{The answer box}

    \subsection*{How to generate/use the answer box.}

        The answer box is the result of the \verb|\answer| command, and is the bread and butter of the math aspect of Ximera. It has a number of optional arguments (covered below) and one required arguemnt - the content author's answer. Thus if you want to provide an open-answer input for students and the correct answer for that spot is $x^2 + 3x + 1$ you would type \verb|\answer{x^2 + 3x + 1}|. Note that the answer box must be put in mathmode. The previous problem would then look like:
        \begin{explanation}
            The correct answer is $x^2 + 3x + 1$: $\answer{x^2 + 3x + 1}$.
        \end{explanation}
        
    \subsection*{How the answer box validates an answer}
        
        It uses more than a dozen algorithms to check if the student's provided answer is mathematically equivalent to the content author's provided answer. Most of these are used as shortcuts to allow faster authentication - for example, if the two answers are exact string matches it will immediately be marked as correct. But if the validation fails (returns false) it proceeds to the next validation method. If every validation method returns false/incorrect, \textit{then} the answer is marked wrong. If \textit{any} of the validators returns ``correct'', then the answer is marked correct. So the multiple validation techniques are more of a sieve than different validation aspects.

        Most noteworthy is that the \textit{final} validator in this sieve is computationally intensive (hence we want to try and do 'quick' checks first) but it is exceptionally powerful. It uses the identity theorem from complex analysis - which essentially says that, if the two functions agree on a subset of $\mathbb{C}$ with an accumulation point, then they are the same function.
        
        In this case, Ximera picks a few hundred points just off the real axis in the complex plane that mimic a converging sequence in some sense (a sequence whose distance between points is strictly decreasing), and tests the student given answer and the author-given answer. If they agree (up to machine precision) on more than 98\% of those points, it declares them equal. The nature of machine precision requires the 98\%, but essentially, if Ximera marks it correct due to this algorithm, it is \textit{exceptionally} unlikely that the answer isn't a correct answer - especially at the student level we use Ximera (i.e. calc sequence and below).
        
        


    \subsection*{Optional Arguments}
    
        There are also a number of optional arguments/key-values that can be provided to the \verb|\answer| command, for a variety of effects. Most of these are ``key-value'' pairs, meaning that they have the form ``key=value''. I outline the key names below and will give examples of values one might use for the given key.


        \begin{description}
            \item[tolerance:] The Tolerance key-value sets a $\pm$ value on the author's (numeric) answer that will be accepted from a student. For example, the answer box \verb|\answer[tolerance=0.1]{\pi}| would accept anything in the range of $[\pi-0.1,\pi+0.1]$ as correct. This was primarily intended to be used for graphical applications, to allow for ``fuzzy answers'' on a graph.
            
            \item[validator:] This key-value allows you to load a validator in-line from a prior javascript environment. It's important to note here that \textbf{this specific validator call} actually uses the Ximera validation system, and not the generic javascript/python call. As a result, it is \textit{far} superior to the validator environment, but it also only works from a single answer box, which makes it difficult to use for more complex problems. 
            
            In practice, you would use a javascript environment to define a js function that returns some kind of true/false type value first. Once you have that 'validateFunc' function, you then call it in the answer box as \verb|\answer[validator=validateFunc]{answer-here}| and the validation of the student answer will be entirely decided by the validateFunc you wrote earlier - which may or may not use the 'answer-here' content-author's answer.
            
            \item[id:] This is used to set an id for the student's answer, which can be called by a validator or some other code. So something like \verb|\answer[id=x]{}| would take whatever the student input into that answer box and assign it as the value of the python variable ``x'', which could be called in a validator environment or some other code somewhere. I haven't played around with this much \textit{other} than in custom validator environments.
            
            \item[format:] The Format key-value can change what format the answer box expects to receive (and thus how to validate). For example, using the command \verb|\answer[format=string]{test}| let's the answer box know that it is evaluating the student answer as a direct string, rather than an equation. As a result, it would only accept the \textit{string} `` test ``. Without the ``format=string'' key-value, any permutation of the letters t, e, s, and t would be accepted as they would be interpreted as variables - thus ``$et^2s$'' would have been accepted. 
            
            Currently, the only value I remember for this key is the ``string'' value, but I know there are others in theory.
            
            \item[given:] This is largely ignored now, but this dictates whether the answer for the answer box is provided in the corresponding pdf that is generated from the Ximera file. This can be used to make a ``teachers edition'' version of the notes, rather than a ``student'' version. For example, the command \verb|\answer[given=true]{\pi}| would display $\pi$ in the pdf, whereas \verb|\answer[given=false]{\pi| would just show a question mark in the box. Note that given=false is the default behavior.
            
        \end{description}
    

    \subsection*{Examples}
    
        Here is an example using a number of different optional arguments.
        
        \begin{problem}
          Type $2$. $\answer{2}$, $\answer[given]{2}$, $\answer[tolerance=.2]{\frac{1}{2}}$,  $\int_a^b f(x) \ dx$,
          \[
          -\frac{1}{\frac{12}{1}}\sum_{n=1}^{\answer{\infty}} n
          \]
        \end{problem}
    
    \subsection*{Best Practices}
    
        \subsubsection*{tolerance optional argument}
            The answer box allows for a tolerance level as stated above. Most often one thinks to use this as a ``round to the nearest hundreth'' or similar style problem, but Ximera supports comparison with machine-level percision (meaning that you can use irrational numbers like $\sqrt{2}$, $\pi$ and $e$ and students would have to enter in far more decimal places than is plausible before they would get a correct answer without actually using the irrational value itself). 
            
            A more pedagogically useful place to use tolerance is in situations where there is an actual lack of precision. For example, if you are asking students for a coordinate on a graph, this can allow you to take in a range of values for each coordinate. 
            
        \subsubsection*{tolerance optional argument pt 2:} 
            One of the issues with using tolerance is that a student can have an answer marked correct, even if it is slightly off from the expected value. This is normally not a big deal, but if there are followup calculations or problems, this small error can compound until - despite doing all the steps correctly, the student eventually gets a value whose error compounded big enough that it is now no longer within the provided tolerance.
            
            To fix this, a content author can try to use progressively larger tolerances, but that is really just making the problem worse and losing the fidelity of what it means to get the problem ``correct''. Alternatively, an author can use the ``id'' optional argument to then call the \textit{students value} via a validator optional argument in future answer boxes, to calculate the exact expected value after a sequence of algebraic steps from the student's answer, rather than the original ``expected'' value. This allows you to maintain a relatively tight tolerance value throughout a sequence of nested problems, without having students compound a small error until their answer is rejected and generating confusion. For more on how to do this, see the validator section. 
        
        \subsubsection*{string formats}
            Although Ximera supports string format authentication, this is such a remarkably unforgiving authentication check. Literally any deviation from the exact sequence of provided characters will be rejected - including things like capitalization differences, extra whitespace, and different character codes. For example, if you have an answer box that looks like: \verb|\answer[format=string]{ab+c}\verb|, the following would all be rejected answers due to the string format:
            \begin{itemize}
                \item ``ab + c'' [rejected due to the space between the plus sign and the letters]
                \item ``AB+C'' [rejected due to the capital letters rather than lower case]
                \item ``abc+ `` [rejected due to the extra space at the end of the answer]
                \item ``abc+'' [rejected because it is using a pasted-in plus sign that is technically a hex character code rather than a unicode character code - even though they look exactly the same \textit{and} compile to pdf exactly the same].
            \end{itemize}
            For this reason, it is generally a good idea to avoid using the format=string optional argument unless you really do need the student to enter in the \textit{exact} sequence of characters in order to make a point. Often if you are just trying to get a ``yes'' or a ``no'', you can accept this without a format=string as there isn't really a pedagogical impact to the letters being accepted out of order (it's not like the student really things that, if the answer ``yse'' is accepted, that ``yes'' was not actually the correct letter, they will just assume it accepted the typo).

        \subsubsection*{Answer Box Size}
            If you are in a situation where you want a student to solve for a portion of an expression or formula, and thus you need to put parentheses around the answer box, keep in mind that normal parentheses won't actually be big enough to go around the answer box and thus it will look kind of goofy. Instead use the \verb|\left(\answer{}$\right)$\verb| style to force the parentheses to grow to the appropriate height. For a visual of what we mean, consider the following - first just parentheses, the second with the recommended method:
            \[
                (\answer{4}) \qquad \left(\answer{4}\right)
            \]
        
        \subsubsection*{Answer Box Size pt 2}
            Also keep in mind that the answer box is a (fairly set) size, which means that it can be hard to tell when it is suppose to be in the place of something small - for example a subscript or superscript. For example:
            \[
                \answer{64}x^{\answer{2}} = 64x^2
            \]
            If you are looking closely, and if you have both levels of answer boxes to compare and see the difference in alignment, you can tell that the first box is clearly a coefficient and the second should be some kind of exponent. But very often students won't catch that - especially if there is only the one answer box - at first glance, so it is helpful to keep this in mind when you are designing problems. Some kind of contextual clue, prompt, or even a hint or feedback to inform the student that the answer box is expecting an exponent, or subscript, or whatever it is, can be extremely helpful and remove considerable frustration on the part of the student.

    
    \subsection*{Potential Pitfalls and Problems}    
    
        \subsubsection*{Ximera Errs on the side of the student}
        
            Generally, answer validation sides with the student. In most cases this philosophy isn't relevant, but it is noteworthy here because this means that if the \verb|\answer| command can't understand your provided answer, then it will default to mark \textit{any} answer from the student correct. In practice, if you find that you have an answer box that is marking anything/everything correct, this is probably because there is some kind of typo or character in the answer box that it cannot understand.
    
        \subsubsection*{Answer Box can be a little too good...} 
            
            Although the validation technique is impressively comprehensive in practice, that can be problematic when you want the answer in a specific \textit{form}. For example, if Ximera will take any function that is mathematically equivalent to the form the author provided, it becomes very difficult to validate whether or not the student's answer is, say, factored. You can often work around this problem with clever question design, or by writing a custom validator to capture the specific property or aspect that you are trying to assess (as it is less important to only validate the equality, and more important to validate the form - often in addition to the equality, of the answer).
            
        \subsubsection*{Ximera is still running on a computer}
            
            Also remember that Ximera is still running on a computer, which means it doesn't \textit{truly} understand real numbers. Indeed, it only really understands computable numbers (which we won't delve into here) but the short version, is that anything that is beyond machine-precision can't really be assessed accurately by a machine. For example, the following answer box is expecting the answer $2x$, but try inputting $2x+2^{-40}$ and see what happens. 
            \begin{explanation}
                $\answer{2x}$. 
            \end{explanation}
            Ximera accepts it as correct, even though we obviously know that's not true. But that's because $2^{-40}$ is a number that is too small for machine precision - so to the computer \textit{running} Ximera, it is effectively indistinguishable from zero. This is just an inherent restriction based on the machine running Ximera, rather than Ximera itself - but it can be important to be aware of, since it can crop up in other ways. For example, if you have something like $e^{-200|x|^{100}}$, then almost regardless of what number Ximera picks in the complex plane, chances are excellent that the result would be too small for machine precision to detect - meaning that the entire \textit{function} is indistinguishable from the constant zero. This can crop up as a result of integrating or differentiating weird functions that end up having extra terms that it is important for students to realize exist (e.g. you want to verify they used the chain rule and not just the base derivative rule) but the ``extra part'' doesn't seem to matter to Ximera - this is typically an issue of machine precision.
            
        
        \subsubsection*{Answers are exposed via right-click}
            
            For accessibility reasons (like screen readers and the like) the library that powers the answer box has a bunch of features bundled in, one of which is the ability to right-click and get different formatting of its contents. Unfortunately, if you right-click and go to ``show math as'' and then ``tex commands'' you can very easily see the contents of the answer command that was used in the answer file. This information isn't necessary for accessibility (after all, the answer box is suppose to contain the actual content supplied by the student, and doesn't have any content the student needs to know about) but it's tied to the behavior of the underlying library that supports \textit{all} the rendered math on the page, so we can't remove it without killing \textit{all} accessibility.
            
            However, it is easily countered in the specific case of the answer box, which we detail in the \href{https://xronos.clas.ufl.edu/examples/exampleCore/supplemental/hiddenAnswers}{how to hide answers} section of the documentation, for those that are interested.
        
    
\end{document}