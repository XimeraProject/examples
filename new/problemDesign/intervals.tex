\documentclass{ximera}
\title{Interval Answer Formats}


\begin{document}
\begin{abstract}
How to get students to input an interval as an answer.
\end{abstract}
\maketitle

This is an example page to show how to take in intervals as answers to a problem. There are a number of options, each one demonstrated below with the corresponding code printed below the answer box/section that you can copy/paste into your own relevant tex file in order to modify and use.


\begin{problem}

    This first is the easiest but least difficult for a student to answer; which is just straight up multiple choice.
    
    We want the interval: $[a,b)$.
    
    \begin{multipleChoice}
        \choice{$(a,b)$}
        \choice{$(a,b]$}
        \choice[correct]{$[a,b)$}
        \choice{$[a,b]$}
    \end{multipleChoice}
    
    The code for the multiple choice answer block is the following:
    \begin{verbatim}
        \begin{multipleChoice}
            \choice{$(a,b)$}
            \choice{$(a,b]$}
            \choice[correct]{$[a,b)$}
            \choice{$[a,b]$}
        \end{multipleChoice}
    \end{verbatim}
    Note that you can add additional \verb|\choice{AnswerOptions}| to include more possible answer options; say if you wanted to add other values than $a$ or $b$ as possible endpoints.
\end{problem}


\begin{problem}
    If you want to force students to determine if an endpoint is included or not (for example, if you are introducing the difference between brackets and parentheses, or doing an interval of convergence problem) you can use a mixture of open answer and multiple choice style dropdown option to build a multi-part answer. 
    
    We want the interval: $[a,b)$
    
    \wordChoice{
        \choice{$($}
        \choice[correct]{$[$}
        \choice{$)$}
        \choice{$]$}
        }
    $\answer{a}$ , $\answer{b}$
    \wordChoice{
        \choice{$($}
        \choice{$[$}
        \choice[correct]{$)$}
        \choice{$]$}
        }
        
    You can reproduce this answer with the following:
    \begin{verbatim}
        \wordChoice{
            \choice{$($}
            \choice[correct]{$[$}
            \choice{$)$}
            \choice{$]$}
            }
        $\answer{a}$ , $\answer{b}$
        \wordChoice{
            \choice{$($}
            \choice{$[$}
            \choice[correct]{$)$}
            \choice{$]$}
            }
    \end{verbatim}
    Be careful to note where the mathmode delimiters are being used; the \verb|\wordChoice{}| command needs to \textit{not} be in mathmode, but \verb|\answer{}| \textit{must} be in mathmode. 
    
%    Furthermore, if you are concerned about students doing this piecemeal to determine the right answer, you can fold the entire answer in a validator environment to force the answers to be checked all at once, rather than one at a time as we demonstrate next:

\end{problem}


\begin{problem}
    Finally, you can just give a number of answer blanks for your students:
    
    We want the interval $[a,b)$.
    
    $\answer[format=string]{[}
    \answer{a}$, 
    $\answer{b} 
    \answer[format=string]{)}$
    
    You can reproduce this with:
    \begin{verbatim}
        $\answer[format=string]{[}
        \answer{a}$, 
        $\answer{b} 
        \answer[format=string]{)}$
    \end{verbatim}
    
    Notice the 'format=string' line in the answer boxes that contain the bracket or parentheses. This is necessary because these are otherwise special characters that will ruin the answer box validation process. This is also why it is \textit{absolutely necessary} to separate any special characters (aka the brackets, braces, and/or parentheses) into their own answer box compared to any other content. 

    
\end{problem}




\end{document}