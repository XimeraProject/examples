\documentclass{ximera}

\outcome{Theorem environments.}

%% You can put user macros here
%% However, you cannot make new environments

\graphicspath{
{./}
{graphicsAndVideos/}
}

\usepackage{tikz}
\usepackage{tkz-euclide}

\newcommand{\licenseAcknowledgement}{Licensed under Creative Commons 4.0}


\newcommand{\dfn}{\textbf}

\pgfplotsset{compat=1.7} % prevents compile error.

\tikzstyle geometryDiagrams=[ultra thick,color=blue!50!black]

\usepackage{lipsum}

\newcommand{\ximera}{XiMeRa}
%\def\d{\,d}
\renewcommand{\d}{\,d}


\newenvironment{MM}{\begin{remark}}{\end{remark}}


%\newtheorem*{MM}{Metacognitive Moment}
%\ximerizedEnvironment{MM}


%% FOR XIMERA CLS %%%%%%%%%%%%%%%%%%%%%%%%%%%%%%%

\renewcommand{\link}[2][]{\href{#2}{#1}% the optional argument,
                                       % a name for the link as a clickable link
\ifthenelse{\equal{#1}{}}% if no optional argument, then 
{\url{#2}}% just print the name
{\footnote{See #1 at \url{#2}}}}% include the link address as a footnote

%%%%%%%%%%%%%%%%%%%%%%%%%%%%%%%%%%%%%%%%%%%%%%%%%%




%% \makeatletter%% note could be made fancy with if then tech above.
%% %\setcounter{secnumdepth}{2}
%% \def\@othersection\section
%% \def\@othersubsection\subsection
%% \newcommand{\sectionstyle}{\def\section{\@othersection}\def\maketitle{\newpage \noindent {\bf A pretitle:}\\[-3em]\section\@title}\def\section{\subsection}}
%% \newcommand{\chapterstyle}{\def\maketitle{\newpage \noindent {\bf A pretitle:}\\[-3em]\chapter\@title}}
%% \makeatother

\author{Bart Snapp \and Rodney Austin}

\title{Theorem-like environments}

\begin{document}
\begin{abstract}
  Examples of the theorem environments.
\end{abstract}
\maketitle

A user defined theorem: 
\begin{MM}
 \lipsum[1][1-3]
\end{MM}

But, \ximera\ provides a number of theorm-like environments.

\begin{theorem}
 \lipsum[1][1-3]
\end{theorem}


\begin{theorem}[My theorem]
  \lipsum[1][1-3]
\end{theorem}

\begin{algorithm}
  \lipsum[1][1-3]
\end{algorithm}

\begin{axiom}
  \lipsum[1][1-3]
\end{axiom}

\begin{claim}
  \lipsum[1][1-3]
\end{claim}

\begin{conclusion}
  \lipsum[1][1-3]
\end{conclusion}

\begin{condition}
  \lipsum[1][1-3]
\end{condition}

\begin{conjecture}
  \lipsum[1][1-3]
\end{conjecture}

\begin{corollary}
  \lipsum[1][1-3]
\end{corollary}

\begin{criterion}
  \lipsum[1][1-3]
\end{criterion}

\begin{definition}
  \lipsum[1][1-3]
\end{definition}

\begin{example}
  \lipsum[1][1-3]
\end{example}

\begin{explanation}
  \lipsum[1][1-3]
\end{explanation}

\begin{fact}
  \lipsum[1][1-3]
\end{fact}

\begin{formula}
  \lipsum[1][1-3]
\end{formula}

\begin{idea}
  \lipsum[1][1-3]
\end{idea}

\begin{lemma}
  \lipsum[1][1-3]
\end{lemma}

\begin{model}
  \lipsum[1][1-3]
\end{model}

\begin{notation}
  \lipsum[1][1-3]
\end{notation}

\begin{observation}
  \lipsum[1][1-3]
\end{observation}

\begin{paradox}
  \lipsum[1][1-3]
\end{paradox}

\begin{procedure}
  \lipsum[1][1-3]
\end{procedure}

\begin{proposition}
  \lipsum[1][1-3]
\end{proposition}

\begin{remark}
  \lipsum[1][1-3]
\end{remark}

\begin{summary}
  \lipsum[1][1-3]
\end{summary}

\begin{template}
  \lipsum[1][1-3]
\end{template}

\begin{warning}
  \lipsum[1][1-3]
\end{warning}


\end{document}
