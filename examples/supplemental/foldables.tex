\documentclass{ximera}
\title{Expandable Content}


\begin{document}
\begin{abstract}
    Demo of collapsible/expandable content.
\end{abstract}
\maketitle
\section*{Expandable and/or Foldable Content}
    \subsection*{How to make expandable/foldable content}
        You can include content that expands out, or collapses back in (i.e. expandable/collapsible boxes, which starts expanded or collapsed). To do this you want to use one of two environments, either the foldable environment or expandable environment.
        
        Thus to have collapsible content that starts compressed, we would type something like:
        \begin{verbatim}
            \begin{expandable}
                This text will be hidden inside an expandable bar with an arrow prompt to the right.
            \end{expandable}
        \end{verbatim}
        
        Which gives:
    
        \begin{expandable}
            This text will be hidden inside an expandable bar with arrow prompt to the right.
        \end{expandable}
            
        And to have collapsible content that starts expanded, we would type something like this:
    
        \begin{verbatim}
            \begin{foldable}
                This text will be visible, but inside a collapsible box with an arrow prompt to the right.
            \end{foldable}
        \end{verbatim}
    
        Which gives:
    
        \begin{foldable}
            This text will be visible, but inside a collapsible box with an arrow prompt to the right.
        \end{foldable}
    
        In theory you could use this to subdivide your content a little more cleanly and provide optional commentary and such that students can collapse/expand as they want. 
        
    \subsection*{Optional Arguments}
        There are currently no optional arguments for these environments.

    \subsection*{(Additional) Examples}
    
        See above. Not sure we need additional examples?
        
    \subsection*{Best Practices}
    
        \subsubsection*{Queue the student to the arrow}
            As mentioned below, the expansion arrow can be easily overlooked, so it is best to put some kind of textual pointer to the student to ensure they see/find the arrow.
        
        \subsubsection*{Make the type of content in expandable/foldable environments consistent in type.}
            It is ideal to treat the expandable/foldable environment as a wrapper around a certain type of content. For example, you may consistently put optional content in these environments, or you may choose to put problem-walkthroughs in these kinds of environments. Or perhaps deep-dives into concepts or formulae derivations.
            
            The point here is that this style of interactive doesn't have a well established role in online educational platforms. As a result, students won't know what to expect when clicking the arrow to expand the content. Building this expectation will help students parse the page contents and lower anxiety around learning, as they will know what to expect when they see these expandable/foldable sections. \textit{What} kind of content you put in these segments, will be a lot less important than \textit{how consistently} the contents are the same kind of content throughout the semester.
    
    \subsection*{Potential Pitfalls and Problems}
    
        \subsubsection*{The Collapsible Arrow is Subtle}
            As you may notice, the arrow that lets you collapse/expand the included content can be pretty small and far to the right, depending on the device and/or window size/resolution the student has, so you may consider prompting them to let them know that some content is expandable so they see the arrow to expand and read info if they are interested. You could also include the environment within other environments to shrink it's size (e.g. inside an explanation environment or something similar) which may help.
    



\end{document}